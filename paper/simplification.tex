\documentclass[sigconf]{acmart}

\usepackage{booktabs} % For formal tables
\usepackage{todonotes}

% Our added packages
\usepackage[ruled]{algorithm2e}
\usepackage{tabu}

%\usepackage{algorithm}
%\usepackage{algorithmic}

\graphicspath{ {figures/} }

% Copyright
%\setcopyright{none}
%\setcopyright{acmcopyright}
%\setcopyright{acmlicensed}
\setcopyright{rightsretained}
%\setcopyright{usgov}
%\setcopyright{usgovmixed}
%\setcopyright{cagov}
%\setcopyright{cagovmixed}


% DOI
\acmDOI{10.475/123_4}

% ISBN
\acmISBN{123-4567-24-567/08/06}

%Conference
\acmConference[GECCO '17]{the Genetic and Evolutionary Computation Conference 2017}{July 15--19, 2017}{Berlin, Germany}
\acmYear{2017}
\copyrightyear{2017}

\acmPrice{15.00}


\begin{document}
\title{Improving Generalization of Evolved Programs through Automatic Simplification}
%\subtitle{Do we want a sub-title?}

\author{Thomas Helmuth}
\orcid{0000-0002-2330-6809}
\affiliation{%
  \institution{Washington and Lee University}
  \city{Lexington} 
  \state{Virginia} 
  \country{USA}
}
\email{helmutht@wlu.edu}

\author{Nicholas Freitag McPhee}
\orcid{0000-0002-6495-2612}
\affiliation{%
  \institution{University of Minnesota, Morris}
  % \streetaddress{600 E. 4th Street}
  \city{Morris} 
  \state{Minnesota} 
  \country{USA}
}
\email{mcphee@morris.umn.edu}

\author{Edward Pantridge}
\affiliation{
	\institution{MassMutual Financial Group}
	\city{Amherst}
	\state{Massachusetts}
	\country{USA}
}
\email{epantridge@massmutal.com}

\author{Lee Spector}
\affiliation{%
	\institution{Hampshire College}
	\city{Amherst}
	\state{Massachusetts}
	\country{USA}
}
\email{lspector@hampshire.edu}

% The default list of authors is too long for headers}
% \renewcommand{\shortauthors}{B. Trovato et. al.}


\begin{abstract}
Just like many areas of machine learning, genetic programming can be prone to overfitting of the training data. As such, the ability to synthesize programs that generalize to unseen data remains an important open problem. We present evidence that smaller programs tend to generalize better, and explore methods of automatically simplifying evolved programs to make them smaller. Automatic simplification uses a simple hill-climbing procedure to remove pieces of a program while ensuring that the resulting program gives the same error vector on the training data as the original program. We show that automatic simplification, previously used for both post-run analysis and as a genetic operator, can be used on evolved programs to significantly improve their generalization rates. We present four new simplification variants and analyze their strengths and weaknesses on a suite of general program synthesis benchmark problems. We conclude that while some simplification methods have slightly more power than others, all methods dramatically decrease program sizes and improve generalization.
	\todo[inline]{We need to do the CCS Concepts thing.}
	\todo[inline]{We need to do the keywords.}
\end{abstract}

%
% The code below should be generated by the tool at
% http://dl.acm.org/ccs.cfm
% Please copy and paste the code instead of the example below. 
%
\begin{CCSXML}
<ccs2012>
 <concept>
  <concept_id>10010520.10010553.10010562</concept_id>
  <concept_desc>Computer systems organization~Embedded systems</concept_desc>
  <concept_significance>500</concept_significance>
 </concept>
 <concept>
  <concept_id>10010520.10010575.10010755</concept_id>
  <concept_desc>Computer systems organization~Redundancy</concept_desc>
  <concept_significance>300</concept_significance>
 </concept>
 <concept>
  <concept_id>10010520.10010553.10010554</concept_id>
  <concept_desc>Computer systems organization~Robotics</concept_desc>
  <concept_significance>100</concept_significance>
 </concept>
 <concept>
  <concept_id>10003033.10003083.10003095</concept_id>
  <concept_desc>Networks~Network reliability</concept_desc>
  <concept_significance>100</concept_significance>
 </concept>
</ccs2012>  
\end{CCSXML}

\ccsdesc[500]{Computer systems organization~Embedded systems}
\ccsdesc[300]{Computer systems organization~Redundancy}
\ccsdesc{Computer systems organization~Robotics}
\ccsdesc[100]{Networks~Network reliability}

\keywords{ACM proceedings, \LaTeX, text tagging}


\maketitle

\section{Introduction}
\label{sec:intro}

\todo[inline]{Motivation and overview.}

Things we need to do somewhere in this paper:
\begin{itemize}
	\item Convince everyone that this is awesome. It would be particularly cool if we can somehow make the case that these ideas are relevant even if people aren't doing stack based work. Can we do that, though? Most (i.e., tree-based)
	GP systems don't allow for things like silencing of nodes or replacing nodes
	with NOOPs because it typically wouldn't be clear how to interpret such a tree.
	\item Cover relevant work. This might require some digging? I bet we cite~\cite{Helmuth:2015:dissertation}, though.
	\item Cover relevant background. This is going to (at least) require explaining PLUSH genomes, the conversion from PLUSH to PUSH, the evaluation of PUSH programs (so we can understand what NOOPs do), and silent genes.
	\item Describe the 5 simplification methods
	\item Experimental setup
	\item Results
	\item Discussion
	\item Future work
	\item Conclude
\end{itemize}

Smaller solutions have other benefits as well: understandability, faster runtimes, etc. But, we won't concentrate on those in this paper.

\section{PUSH, PLUSH, etc.}
\label{sec:push}

\todo[inline]{Explain necessary aspects of PUSH, PLUSH, and the like.}

\section{Automatic Simplification in Push}
\label{sec:simplification}

Automatic simplification of Push programs was first described in work detailing post-run simplification \cite{Spector:2014:GECCOcomp}, but was in use even before then to help with understandability of evolved Push programs. It has also been used as a genetic operator, with the idea of simplification being used as a bloat control operator \cite{Zhan:2014:GECCOcomp}. \todo[inline]{TMH: These sentences are sloppy and should be cleaned up/expanded.}


Types of automatic simplification (EXPLAIN THESE): program, genome, genome-backtracking, genome-noop, genome-backtracking-noop.


\section{Experimental setup}
\label{sec:setup}

\todo[inline]{Describe the setup, problems, etc.}





\section{Results}
\label{sec:results}

\todo[inline]{Describe the results.}

\section{Discussion}
\label{sec:discussion}

\todo[inline]{Talk about what the results mean}

\section{Related Work}
\label{sec:related}

\todo[inline]{Below are things I've (Tom) come across in the past year potentially realted to this work. We'll have to go through and see what's worth citing.}

\subsection{Papers about automatic simplification}

\begin{itemize}

\item
Genprog minimization after run (see 7/24/16)

\item
Field Guide to Genetic Programming: p 64 top: Bahnzaf paper might be precursor to automatic simplification

\item
This paper in neural networks might be related: GECCO 2016 - Identifying Core Functional Networks and Functional Modules within Artificial Neural Networks via Subsets Regression

\item
Differentiate between our work and algebraic simplification (as used in semantic GP), since this CAN change the semantics on inputs not in the training set, where algebraic methods cannot. Also, algebraic methods wouldn't work on general programs that we're evolving

\item
Algebraic Simplification of GP Programs During Evolution (I think this is an actual paper title)

\item
Investigation of simplification threshold and noise level of input data in numerical simplification of genetic programs \cite{Kinzett:2010:cec}

\item
Smaller networks in neural nets: https://push-language.hampshire.edu/t/gecco-2017-simplification-for-generalization/660/21?u=thelmuth

\end{itemize}

\subsection{Papers about generalization and overfitting in GP}

\begin{itemize}
\item
GECCO 2011 - Variance based Selection to Improve Test Set Performance in Genetic Programming

\item
Controlling overfitting in symbolic regression based on a bias/variance error decomposition


\end{itemize}

\section{Conclusions and future work}
\label{sec:conclusions}

\todo[inline]{Maybe this could/should be two sections, but I bet we won't have room.}

\begin{acks}
  
  \todo[inline]{Put acknowledgements here, including grants.}

\end{acks}


\bibliographystyle{ACM-Reference-Format}
\bibliography{simplification} 

\end{document}
