\section{Introduction}
\label{sec:intro}

\todo[inline]{Motivation and overview.}

Things we need to do somewhere in this paper:
\begin{itemize}
	\item Convince everyone that this is awesome. It would be particularly cool if we can somehow make the case that these ideas are relevant even if people aren't doing stack based work. Can we do that, though? Most (i.e., tree-based)
	GP systems don't allow for things like silencing of nodes or replacing nodes
	with NOOPs because it typically wouldn't be clear how to interpret such a tree.
	\item Cover relevant work. This might require some digging? I bet we cite~\cite{Helmuth:2015:dissertation}, though.
	\item Cover relevant background. This is going to (at least) require explaining PLUSH genomes, the conversion from PLUSH to PUSH, the evaluation of PUSH programs (so we can understand what NOOPs do), and silent genes.
	\item Describe the 5 simplification methods
	\item Experimental setup
	\item Results
	\item Discussion
	\item Future work
	\item Conclude
\end{itemize}

\section{PUSH, PLUSH, etc.}
\label{sec:push}

\todo[inline]{Explain necessary aspects of PUSH, PLUSH, and the like.}

\section{Automatic Simplification in Push}
\label{sec:simplification}

Automatic simplification of Push programs was first described in work detailing post-run simplification \cite{Spector:2014:GECCOcomp}, but was in use even before then to help with understandability of evolved Push programs. It has also been used as a genetic operator, with the idea of simplification being used as a bloat control operator \cite{Zhan:2014:GECCOcomp}. \todo[inline]{TMH: These sentences are sloppy and should be cleaned up/expanded.}


Types of automatic simplification (EXPLAIN THESE): program, genome, genome-backtracking, genome-noop, genome-backtracking-noop.


\section{Experimental setup}
\label{sec:setup}

\todo[inline]{Describe the setup, problems, etc.}





\section{Results}
\label{sec:results}

\todo[inline]{Describe the results.}

\section{Discussion}
\label{sec:discussion}

\todo[inline]{Talk about what the results mean}

\section{Related Work}
\label{sec:related}

\todo[inline]{Below are things I've (Tom) come across in the past year potentially realted to this work. We'll have to go through and see what's worth citing.}

\subsection{Papers about automatic simplification}

\begin{itemize}

\item
Genprog minimization after run (see 7/24/16)

\item
Field Guide to Genetic Programming: p 64 top: Bahnzaf paper might be precursor to automatic simplification

\item
This paper in neural networks might be related: GECCO 2016 - Identifying Core Functional Networks and Functional Modules within Artificial Neural Networks via Subsets Regression

\item
Differentiate between our work and algebraic simplification (as used in semantic GP), since this CAN change the semantics on inputs not in the training set, where algebraic methods cannot. Also, algebraic methods wouldn't work on general programs that we're evolving

\item
Algebraic Simplification of GP Programs During Evolution (I think this is an actual paper title)

\item
Investigation of simplification threshold and noise level of input data in numerical simplification of genetic programs \cite{Kinzett:2010:cec}

\item
Smaller networks in neural nets: https://push-language.hampshire.edu/t/gecco-2017-simplification-for-generalization/660/21?u=thelmuth

\end{itemize}

\subsection{Papers about generalization and overfitting in GP}

\begin{itemize}
\item
GECCO 2011 - Variance based Selection to Improve Test Set Performance in Genetic Programming

\item
Controlling overfitting in symbolic regression based on a bias/variance error decomposition


\end{itemize}

\section{Conclusions and future work}
\label{sec:conclusions}

\todo[inline]{Maybe this could/should be two sections, but I bet we won't have room.}

\begin{acks}
  
  \todo[inline]{Put acknowledgements here, including grants.}

\end{acks}
