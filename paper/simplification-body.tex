\section{Introduction}
\label{sec:intro}

\todo[inline]{Motivation and overview.}

Things we need to do somewhere in this paper:
\begin{itemize}
	\item Convince everyone that this is awesome. It would be particularly cool if we can somehow make the case that these ideas are relevant even if people aren't doing stack based work. Can we do that, though? Most (i.e., tree-based)
	GP systems don't allow for things like silencing of nodes or replacing nodes
	with NOOPs because it typically wouldn't be clear how to interpret such a tree.
	\item Cover relevant work. This might require some digging? I bet we cite~\cite{Helmuth:2015:dissertation}, though.
	\item Cover relevant background. This is going to (at least) require explaining PLUSH genomes, the conversion from PLUSH to PUSH, the evaluation of PUSH programs (so we can understand what NOOPs do), and silent genes.
	\item Describe the 5 simplification methods
	\item Experimental setup
	\item Results
	\item Discussion
	\item Future work
	\item Conclude
\end{itemize}

\section{PUSH, PLUSH, etc.}
\label{sec:push}

\todo[inline]{Explain necessary aspects of PUSH, PLUSH, and the like.}

\section{Experimental setup}
\label{sec:setup}

\todo[inline]{Describe the setup, problems, etc.}

\section{Results}
\label{sec:results}

\todo[inline]{Describe the results.}

\section{Discussion}
\label{sec:discussion}

\todo[inline]{Talk about what the results mean}

\section{Conclusions and future work}
\label{sec:conclusions}

\todo[inline]{Maybe this could/should be two sections, but I bet we won't have room.}

\begin{acks}
  
  \todo[inline]{Put acknowledgements here, including grants.}

\end{acks}
